\documentclass{IEEEcsmag}

\usepackage[colorlinks,urlcolor=blue,linkcolor=blue,citecolor=blue]{hyperref}
\expandafter\def\expandafter\UrlBreaks\expandafter{\UrlBreaks\do\/\do\*\do\-\do\~\do\'\do\"\do\-}
\usepackage{upmath,color}

\usepackage[spanish]{babel}
%\usepackage[latin1]{inputenc}
\usepackage[utf8]{inputenc}  

\jvol{1}
\jnum{1}
\paper{1}
\jmonth{Noviembre}
\jname{ITICs letters}
\jtitle{Proyectos Integradores}
\pubyear{2023}
\usepackage{cite}
\usepackage{amsmath,amssymb,amsfonts}
\usepackage{algorithmic}
\usepackage{graphicx}
\usepackage{textcomp}
\usepackage{xcolor}
\usepackage{listings}
\usepackage{float}

\newtheorem{theorem}{Theorem}
\newtheorem{lemma}{Lemma}


\setcounter{secnumdepth}{0}

\begin{document}

%imprimir código
\lstnewenvironment{javaCode}[1][]
{\lstset{
    language=Java,
    basicstyle=\scriptsize\ttfamily,
    numbers=none, % Modificado: quitar los números de línea
    keywordstyle=\color{blue},
    commentstyle=\color{gray},
    stringstyle=\color{purple},
    breaklines=true,
    breakatwhitespace=true,
    tabsize=4,
    showspaces=false,
    showstringspaces=false,
    frame=single,
    captionpos=b,
    floatplacement=!h,
    #1
}}
{}


\sptitle{Proyecto Integrador de Primer Semestre}

\title{Software de resolución de problemas de Ingeniería }

\author{Chávez Atanacio Yael Antonio}
\affil{Instituto Tecnológico Superior del Occidente del Estado de Hidalgo, Mixquiahuala, Hgo., 42700, Mexico}

\author{Hernandez Martinez Brayan}
\affil{Instituto Tecnológico Superior del Occidente del Estado de Hidalgo, Mixquiahuala, Hgo., 42700, Mexico}

\author{Cruz Martinez Alejandro}
\affil{Instituto Tecnológico Superior del Occidente del Estado de Hidalgo, Mixquiahuala, Hgo., 42700, Mexico}

%\author{Third Author III}
%\affil{Institute, City, (State), Postal Code, Country}

\markboth{ITSOEH/ITICS/PROYECTO INTEGRADOR PRIMER SEMESTRE}{THEME/FEATURE/DEPARTMENT}

\begin{abstract}
Un resumen (abstract) es un párrafo único que resume los aspectos importantes del manuscrito. A menudo indica si el manuscrito es un informe de un trabajo nuevo, una revisión o una descripción general, o una combinación de ambos. No cite referencias en el resumen. Este tipo de documento debe incluir contenido propiedad de los autores; es decir, no debe contener contenido de otras fuentes, ademas la redacción debe  estar dirigida a un tipo de lector técnico general. Este archivo se encuentra disponible en \href{https://github.com/fcuadrosgithub/integrador-primero.git}{https://github.com/fcuadrosgithub/integrador-primero.git}.
\end{abstract}

\maketitle
\chapteri{L}a introducción debe proporcionar información general (incluidas referencias relevantes) y debe indicar el propósito del manuscrito. En esta sección describa de manera clara y precisa el objetivo del proyecto integrador, la metodología que piensa usar y los resultados obtenidos de manera muy general. Dentro de esta sección puede citar trabajos relevantes de otros si lo cree necesario.

Esta sección debe dar un panorama muy general al lector de cual es el problema a resolver, que metodología utilizó para dar solución al problema y cuales fueron los resultados obtenidos. 

La redacción del manuscrito debe ser en tercera persona y queda estrictamente prohibido el uso de palabras coloquiales o Español informal. En lugar de esto utilice un lenguaje formal que el mayor numero de personas pueda entender.
\clearpage

\section{Resolución Problema 1} 

\subsection{\textbf{Problema 1:}}
Dado un número binario de $n$ bits regresar su equivalente en decimal. 

\subsection{\textbf{Descripción del problema:}}

\subsection{\textbf{Definición de solución:}}

\subsection{\textbf{Diseño de la solución:}}

\subsection{\textbf{Desarrollo de la solución:}}

\subsection{\textbf{Depuración y pruebas:}}
\clearpage

\section{Resolución Problema 2} 
Dada una ecuación cuadrática regresar los valores de las raíces, en caso de que estén sobre el conjunto de números reales, en caso contrario indicar que la solución esta en el conjunto de los números complejos.


\subsection{\textbf{Descripción del problema:}}
El problema planteado se centra en la resolución de una ecuación cuadrática y la determinación de sus posibles soluciones. La ecuación cuadrática tiene la forma general $ax^{2}+bx+c$, donde a, b y c son los coeficientes dados. El objetivo es aplicar la fórmula general:
\begin{equation}
\\
\\ x = \frac{{-b \pm \sqrt{{b^2 - 4ac}}}}{{2a}} \\
\end{equation}
\\
\\Para obtener las soluciones de la ecuación.

Para determinar la naturaleza de las soluciones, se utiliza el discriminante, que está dado por la expresión 
\begin{equation}
 b^2 - 4ac 
\end{equation}
\\Dependiendo del valor del discriminante, se obtienen tres casos posibles:
\begin{itemize}
    \item Si el discriminante > 0 se dan dos soluciones
    \item Si el discriminante = 0 se da una solución
    \item Si el discriminante es < 0 la solución se encuentra en el conjunto de los números complejos. 
\end{itemize}
 Estos casos permiten clasificar las soluciones de la ecuación cuadrática en función de su naturaleza y son determinantes para comprender las propiedades y características de la ecuación en cuestión.


\begin{figure}[H]
    \centering
    \includegraphics[width = 6 cm]{LaTeX/imagenes/discriminant.jpg}
    \caption{Discriminante}
    \label{fig:Discriminante}
\end{figure}


\subsection{\textbf{Definición de solución:}}
Se plantea la resolución de una ecuación cuadrática en la forma 
\begin{equation}
ax^{2}+bx+c
\end{equation}
donde se procede al cálculo del discriminante y a la evaluación de tres condiciones. En caso de que el discriminante sea mayor que cero, se llevan a cabo dos operaciones y se presenta el resultado correspondiente. Si el discriminante es igual a cero, se realiza una operación y se muestra el resultado obtenido. Por otro lado, si el discriminante resulta ser negativo, se informa al usuario que la ecuación en cuestión no cuenta con soluciones reales.

Acto seguido se muestra el diagrama de flujo, el cual es la base del programa.

\begin{figure}[H]
    \centering
    \includegraphics[width = 6 cm]{LaTeX/imagenes/Diagrama2.png}
    \caption{Diagrama de flujo}
    \label{fig:diagrama de flujo}
\end{figure}


\subsection{\textbf{Diseño de la solución:}}
Se plantea la resolución de una ecuación cuadrática en la forma 
\begin{equation}
ax^{2}+bx+c
\end{equation}
donde se procede al cálculo del discriminante y a la evaluación de tres condiciones. En caso de que el discriminante sea mayor que cero, se llevan a cabo dos operaciones y se presenta el resultado correspondiente. Si el discriminante es igual a cero, se realiza una operación y se muestra el resultado obtenido. Por otro lado, si el discriminante resulta ser negativo, se informa al usuario que la ecuación en cuestión no cuenta con soluciones reales.

Acto seguido se muestra el diagrama de flujo, el cual es la base del programa.

\begin{figure}[H]
    \centering
    \includegraphics[width = 6 cm]{LaTeX/imagenes/Diagrama2.png}
    \caption{Diagrama de flujo}
    \label{fig:diagrama de flujo}
\end{figure}


\subsection{\textbf{Desarrollo de la solución:}}
eguidamente se mostrara el programa de java, donde se trabajo con la formula general. Para ello, se ocupan las librería Math y el objeto Scanner.
\begin{javaCode}
    
import java.util.Scanner;

    
        Crear un objeto Scanner para leer la entrada del usuario
        \begin{javaCode}
        Scanner coeficiente = new Scanner(System.in);
        \end{javaCode}
        Solicitar al usuario que ingrese el valor de A
        \begin{javaCode}
        System.out.println("Estimado usuario, por favor ingrese el valor de A: ");
        double a = coeficiente.nextDouble();
        \end{javaCode}
        Solicitar al usuario que ingrese el valor de B
        \begin{javaCode}
        System.out.println("Estimado usuario, por favor ingrese el valor de B: ");
        double b = coeficiente.nextDouble();
        \end{javaCode}
        Solicitar al usuario que ingrese el valor de C
        \begin{javaCode}
        System.out.println("Estimado usuario, por favor ingrese el valor de C: ");
        double c = coeficiente.nextDouble();
        \end{javaCode}
        Cierra el objeto Scanner
        \begin{javaCode}
        coeficiente.close();
        \end{javaCode}
        Calcula el discriminante de la ecuación cuadrática
        \begin{javaCode}
        double discriminante = b * b - 4 * a * c;
        \end{javaCode}
       El objetivo es verificar si el discriminante de una ecuación cuadrática es mayor que cero. Si se cumple esta condición, se determina que la ecuación tiene dos soluciones. Luego, se resuelven las soluciones y se muestra el resultado al usuario.
        \begin{javaCode}
        if (discriminante > 0) {
            double x1 = (-b + Math.sqrt(discriminante)) / (2 * a);
            double x2 = (-b - Math.sqrt(discriminante)) / (2 * a);
       
            System.out.println("Las soluciones son x1 = " + x1 + " y x2 = " + x2);
        }
        \end{javaCode}
        Verificar si el discriminante es igual a 0, si cumple con la condición, realiza la operación donde solo existe una única solución y la muestra al usuario
        \begin{javaCode}
        else if (discriminante == 0) 
           
            double x = -b / (2 * a);

            System.out.println("La solucion es: " + x);
        
        \end{javaCode}
        En caso de que el discriminante sea negativo, se determina que la ecuación cuadrática no tiene soluciones reales. Esta información se comunica al usuario.
        \begin{javaCode}
        else {
            System.out.println("no cuenta con soluciones reales.");
        }
    



\end{javaCode}



\subsection{\textbf{Depuración y pruebas:}}
del mismo. Donde se muestra los diferentes resultados posibles.
\begin{center}
\begin{tabular}{|c|c|c|c|c|}
\hline
No. & A & B & C & Resultado \\
\hline
1 & 19 & 17 & 18 & no cuenta con soluciones reales. \\
\hline
2 & 10 & 15 & 5 & La solución es:$x_1 = -0.5$ y $x_2=-1.0$ \\
\hline
3 & 1 & -2 & 1 & La solución es:$x=1 \\
\hline
4 & 2 & 4 & -6 &La solución es:$x_1 = 1.0$ y $x_2=-3.0$ \\
\hline
5 & 1 & 1 & 1 & No cuenta con soluciones reales. \\
\hline
\end{tabular}
\label{fig: Tabla de ejecución}
\end{center}
\clearpage

\section{Resolución Problema 3}
Dada una circunferencia con centro en el punto
C con coordenadas (x1, y1) y radio r, evaluar si
un punto T con coordenadas (x2, y2) esta dentro
del área de la circunferencia.


\subsection{\textbf{Descripción del problema:}}
El problema implica calcular la distancia de dos puntos 
Dónde el punto C y el punto T pueden ser definidos en el plano cartesiano (X,Y) y R es definidas asía los cuatro ejes, además debemos tomar en cuenta que el punto T se puede encontrar dentro o fuera de la circunferencia 


\subsection{\textbf{Definición de solución:}}
En el primer segmento considerando las variables X1,X2 y Y1,Y2 se plantea la siguiente solución se emplea la formula: distancia:
\\
${Distancia} = \sqrt{{(x_2 - x_1)^2 + (y_2 - y_1)^2}}$,
\\
Con esta formula se calcula la distancia entre los dos puntos  \texttt{(C y T)}.Posterior mente se verifica que el radio ingresado sea positivo, en dado caso de que el radio sea negativo se utilizara la formula: Radio*-1; con esta formula se convierte el radio negativo a positivo,finalmente verificamos que el punto se encuentre dentro de la circunferencia utilizando el radio . 
\\
-Observamos como se plasma en el plano cartesiano utilizando la formula de la distancia
\begin{figure}[H]
    \centering
    \includegraphics[width = 5 cm]{LaTeX/latex-imagenes/Imagen7.png}
    \caption{Gráfica del programa}
    \label{fig:imagen7}
\end{figure}


\subsection{\textbf{Diseño de la solución:}}
\begin{enumerate}[label=\textbf{\arabic*.}]

  \item Considerando las variables X,Y de cada punto calcula la distancia entre los dos puntos utilizando la siguiente formula: {distancia} = \sqrt{{(x_2 - x_1)^2 + (y_2 - y_1)^2}}.

  \item  Considerando que la variable r sea positiva se procedera al siguiente paso, de lo contrario se aplicara la siguiente ecuacion r*-1.
  \item considerando que la distancia sea menor al radio se puede decir que el punto T se encuentra dentro de la circunferencia, de lo contrario se considera que el punto se encuentra fuera de la circunferencia. 
  \\
  A continuación veremos el diagrama de flujo que fue la base para el desarrollo del programa.
\end{enumerate}
\begin{figure}[H]
    \centering
    \includegraphics[width = 8 cm]{LaTeX/latex-imagenes/diagrama.png}
    \caption{Diseño del programa}
    \label{fig:imagen7}
\end{figure}


\subsection{\textbf{Desarrollo de la solución:}}
A continuación, se muestra el código en Java para calcular la distancia entre dos puntos y verificar si se encuentran dentro de una circunferencia:


  \\
  \\
  
        Primero se solicita al usuario las coordenadas del punto C (x1,y1)
       \begin{javaCode}
        Scanner dato = new Scanner(System.in);
        System.out.print("Ingrese las coordenada del punto C primero x1: ");
        int x1 = dato.nextInt();
        System.out.print("Ingrese las coordenada del punto C despues  y1: ");
        int y1 = dato.nextInt();
         \end{javaCode}
         se solicita el radio de la circunferencia
       
        \begin{javaCode}
        System.out.println("Ingrese el radio de la circunferencia: ");
        int radio = dato.nextInt();
         \end{javaCode}
         
        solicitamos al usuario las coordenadas del punto T (x2,y2)
       \begin{javaCode}
        System.out.print("Ingrese las coordenada del punto T X2: ");
        int x2 = dato.nextInt();
        System.out.print("Ingrese las coordenada del punto T Y2: ");
        int y2 = dato.nextInt();
        \end{javaCode}
         Se realiza el cálculo  la distancia entre el punto C y el punto T con ayuda de la formula antes mencionada.
        \begin{javaCode}
        double distancia = Math.sqrt((x2 - x1) * (x2 - x1) + (y2 - y1) * (y2 - y1));
        \end{javaCode}
         Se imprime el resultado del calculo de la distancia.
          \begin{javaCode}
         System.out.println("la distancia es de "+distancia);
          \end{javaCode}
       
        Se verifica si el radio es positivo o negativo, si es negativo lo pasamos a positivo
         \begin{javaCode}
        int radio2=radio*-1;
\end{javaCode}
         Se calcula, si la distancia resulta menor o igual al radio entonces el punto T se encontrara dentro de la circunferencia, de lo contrario el punto T se encontrara fuera de la circunferencia .    
       
        \begin{javaCode}
                    if(radio<0){
            
        
        int radio2=radio*-1;
        if (distancia <= radio2) {
            System.out.println("El punto T está dentro de la circunferencia");
        } else {
            System.out.println("El punto T no está dentro de la circunferencia");
        }
        }
        
       if(radio>0){
           
       
            if (distancia <= radio) {
            System.out.println("El punto T está dentro de la circunferencia");
        } else {
            System.out.println("El punto T no está dentro de la circunferencia");
        }
    }
    
    
    }
    
}
    \end{javaCode}

    
\subsection{\textbf{Depuración y pruebas:}}
Se realizaron 5 pruebas al programa para validar su funcionamiento, a continuación veremos los siguientes resultados 



   \begin{figure}[H]
    \centering
    \includegraphics[width = 11 cm]{LaTeX/latex-imagenes/tabla de pruebas.png}
    \caption{Tabla de pruebas}
    \label{fig:Grafica de la distancia de dos puntos }
\end{figure}
\clearpage

\section{Resolución Problema 4}
Dado un numero decimal de $n$ bits regresar su equivalente en binario.


\subsection{\textbf{Descripción del problema:}}
Consiste en diseñar un enfoque matemático que dado un número decimal, facilite su conversión a su equivalente en binario. Esta problemática requiere el uso de principios fundamentales para abordar eficazmente el desafío en cuestión. Desarrollando un código en Java que emplee una solución formal y eficiente que realice la conversión precisa de números decimales a binarios, incorporando conceptos algebraicos y numéricos relevantes para lograr una comprensión clara y estructurada del proceso.


\subsection{\textbf{Definición de solución:}}
La solución propuesta para el problema se basa en la aplicación de un proceso matemático eficiente y estructurado donde se emplean tres funciones fundamentales para lograr la conversión precisa.

En primer lugar, la función \texttt{divisiónSucesiva} se encarga de realizar la división sucesiva por 2 del número decimal, registrando los residuos en el proceso. Después, la función \texttt{registroResiduos} utiliza estos residuos para construir la representación binaria del número. Finalmente, la función \texttt{obtenerBinario} organiza los resultados para la representación binaria.
Para convertir un número decimal a binario mediante la división sucesiva por 2, puedes utilizar la siguiente ecuación.
\ignorespaces
residuo $m$ - cociente $n-1$ \mod 2.
    \label{eqn:rectaPendiente}

Donde $\text{cociente}_n$ es el cociente en la enésima división, y $\text{cociente}_{n-1}$ es el cociente de la división anterior. 

\begin{figure}[H]
    \centering
    \includegraphics[width = 6 cm]{Latex-imágenes/numeroB.png}
    \caption{Calcula número Binario}
    \label{fig:Calcula número Binario}
\end{figure}


\subsection{\textbf{Diseño de la solución:}}
Para poder entender como debemos empezar el código tenemos que empezar por diseñar el diagrama de flujo 

\begin{figure}[h!]
    \centering
    \includegraphics[width = 6 cm]{Latex-imágenes/DiagramaBinario.jpeg}
    \caption{Diagrama de flujo para la solución}
    \label{fig: Calcular número Binario}
\end{figure}


\subsection{\textbf{Desarrollo de la solución:}}
En este fragmento, se utiliza un Scanner para obtener la entrada del usuario, que es un número decimal entero, ya sea positivo o negativo
\begin{javaCode}

Scanner in = new Scanner(System.in);
System.out.println("Ingrese el número decimal entero positivo o negativo:");
int numeroDecimal = in.nextInt();
in.close();
        
\end{javaCode}

Se inicializa una cadena binario para almacenar la representación binaria y se crea una variable numero que se utilizará para trabajar con el valor absoluto del número decimal.

\begin{javaCode}
    String binario = "";
int numero = numeroDecimal;
\end{javaCode}

Aquí si el número es negativo, se multiplica por -1 para obtener su valor absoluto

\begin{javaCode}
    //Numero Negativo
    if (numero < 0) {
    numero *= -1;
}
\end{javaCode}
//Conversión a Binario (Parte Positiva)
Se realiza la conversión a binario utilizando un bucle while.\\
\begin{javaCode}
    if (numero > 0) {
    while (numero != 0) {
        int residuo = numero % 2;
        binario = residuo + binario;
        numero = numero / 2;
    }
}   
\end{javaCode}

Si el número original es cero, se establece la representación binaria como "0".

\begin{javaCode}
   //Manejo de caso "0"
   ( 
        if (numero == 0) {
    binario = "0";
}
\end{javaCode}

Si el número original es negativo, se busca el índice del último '1' en la cadena binaria invertida.
Se invierte y modifica la cadena binaria para cambiar los '0' por '1' y viceversa.
Se agrega el signo negativo al resultado final.

\begin{javaCode}
// Inversión y Modificación de la Cadena Binaria (Parte Negativa)
   if (numeroDecimal < 0) {
    // Obtener el índice del último '1' en la cadena binaria invertida
    int indiceUltimoUno = 0;
    for (int i = binario.length() - 1; i >= 0; i--) {
        if (binario.charAt(i) == '1') {
            indiceUltimoUno = i;
            break;
        }
    }

    // Invertir y modificar la cadena binaria para números negativos
    String binarioInvertido = binario.substring(0, indiceUltimoUno);
    binarioInvertido = binarioInvertido.replaceAll("1", "#");
    binarioInvertido = binarioInvertido.replaceAll("0", "1");
    binarioInvertido = binarioInvertido.replaceAll("#", "0");

    // Agregar el signo negativo al resultado
    binarioInvertido = "-" + binarioInvertido + binario.substring(indiceUltimoUno);
    binario = binarioInvertido;
}
\end{javaCode}
Finalmente se imprime el resultado final en formato binario.
(positivo o negativo) y maneja adecuadamente los casos especiales, como cero y números negativos.

\begin{javaCode}
   //Mostrar el resultado
   System.out.println("\nEl número convertido a binario es: " + binario);
\end{javaCode}


\subsection{\textbf{Depuración y pruebas:}}
Se realizaron las pruebas de decimal a binario para encontrar errores
\begin{tabular}{|c|c|}
\hline
\textbf{Número Decimal} & \textbf{Número Binario} \\
\hline
1 & 01 \\
\hline
5 & 01 \\
\hline
3 & 0 \\
\hline
\end{tabular}
\\
\\
\clearpage

\section{Resolución Problema 5}
Dado un numero binario de $n$ bits regresa su equivalente a decimal.


\subsection{\textbf{Descripción del problema:}}

El problema consiste en convertir un número binario de n bits a su equivalente en decimal. Un número binario está compuesto únicamente por los dígitos 0 y 1, mientras que un número decimal utiliza la base 10 y está compuesto por los dígitos del 0 al 9.

La tarea es tomar el número binario de entrada, que tiene una longitud específica de n bits, y convertirlo a su representación decimal correspondiente. Esto implica realizar un cálculo que involucra la suma ponderada de los dígitos binarios, donde cada dígito se multiplica por la potencia de 2 correspondiente a su posición en el número binario.

\subsection{\textbf{Definición de solución:}}
La solución para convertir un número binario de n bits a su equivalente en decimal consiste en recorrer cada bit del número binario, multiplicar su valor por 2 elevado a la potencia correspondiente según su posición, y sumar los resultados. Al finalizar el recorrido, el valor obtenido será el equivalente decimal del número binario. Es importante verificar la validez del número binario y asegurarse de que esté compuesto únicamente por 0 y 1. El resultado se muestra como el equivalente decimal del número binario.


\subsection{\textbf{Diseño de la solución:}}
El programa solicita al usuario que ingrese un número binario de n bits. Se asegura de que el número ingresado consista únicamente en dígitos binarios 0 y 1. Si se detecta algún dígito decimal, el programa no funcionará.

Luego, se realiza la conversión del número binario ingresado a su equivalente decimal utilizando la fórmula de posición y peso explicada anteriormente.

Finalmente, el programa imprime el resultado de la conversión, es decir, el número decimal equivalente al número binario ingresado por el usuario.


\begin{figure}[H]
    \centering
    \includegraphics[width=6cm]{Latex-imágenes/ALGORITMO.png}
    \caption{Diagrama de flujo usado como base .}
\end{figure}


\subsection{\textbf{Desarrollo de la solución:}}
El desarrollo del código del programa en Java para la conversión de números binarios a decimales:

\begin{javaCode}[style=javaStyle]
import java.util.Scanner;
import java.math.BigInteger;

public class NewClass {
    public static void main(String[] args) {
        Scanner bin = new Scanner(System.in);
        boolean continuar = true;
\end{javaCode}
En esta paso en el programa son colocadas las variables $Scanner$ $BigInteger$
las cuales ayudaran al correr el programa

Se solicita al usuario ingresar un numero binario
\begin{javaCode}
// Bucle que permite al usuario ingresar números binarios hasta que ingrese 'x'

        while (continuar) {
        
            System.out.print("Ingrese un número en binario (o 'x' para terminar ): ")
            
            String entrada = bin.nextLine();
\end{javaCode}

 La variable $while$ es utilizada como un "mientras" esta mantiene un bucle regresando a pedirnos un numero binario para su conversión a binario

\begin{javaCode}
                 // Si se ingresa 'x', el bucle se detiene;
            if (entrada.equalsIgnoreCase("x")) {
                continuar = false;
                continue;
            }
\end{javaCode}

En esta línea, se declara una variable llamada decimal del tipo BigInteger. Se utiliza para almacenar el valor decimal resultante después de convertir el número binario ingresado. La función convertirBinarioADecimal se llama con el argumento entrada, que es el número binario ingresado por el usuario.

\begin{javaCode}
 // llama al método convertirBinarioADecimal para convertir el número binario ingresado a decimal
 
            BigInteger decimal = convertirBinarioADecimal(entrada);
            
\end{javaCode}

 Esta línea imprime en la consola el mensaje "El número en decimal es: " seguido del valor almacenado en la variable decimal. Es decir, muestra el resultado de la conversión en formato decimal.

\begin{javaCode}
System.out.println("El número en decimal es: " + decimal);

\end{javaCode}

Estas líneas definen el método convertirBinarioADecimal, que toma un argumento de tipo String llamado binario. El método crea un nuevo objeto BigInteger utilizando el constructor que toma dos argumentos: el número binario (binario) y la base (2) que se utiliza para interpretar el número binario y convertirlo a decimal. 

\begin{javaCode}
lic static BigInteger convertirBinarioADecimal(String binario) {
        return new BigInteger(binario, 2);

\end{javaCode}

\text La fórmula se basa en el hecho de que cada posición en un número binario representa una potencia de 2. El dígito en la posición más a la derecha tiene un peso de \(2^0\), el siguiente dígito tiene un peso de \(2^1\), el siguiente tiene un peso de \(2^2\), y así sucesivamente.
\space

\begin{equation}

M = D0 * 2^0 + D1 * 2^1 + D2 * 2^2

\end{equation}


\subsection{\textbf{Depuración y pruebas:}}
En esta tabla se dasarrollan una seria de pruebas, 
\begin{figure}[H]
    \centering
    \includegraphics[width=10cm]{Latex-imágenes/TABLA DE PRUEBAS.png}
    \caption{Prueba de escritorio.}
\end{figure}

\clearpage

\input{Ex6}
\clearpage


\section{CONCLUSION}
El manuscrito debe incluir direcciones futuras de la investigación. Se recomienda encarecidamente a los autores que no hagan referencia a varias figuras o tablas en la conclusión; estos deben mencionarse en el cuerpo del artículo.
\vspace*{-8pt}


\section{AGRADECIMIENTOS}
Esta sección es opcional. Si los autores creen necesario agradecer a alguien por haber aportado al desarrollo de su proyecto integrador de alguna u otra forma, esta sección esta destinada para esto.


\def\refname{REFERENCES}

\begin{thebibliography}{1}
    \sección{Referencias del problema 1}
    \\
    \bibitem{referencia citada}
    1-Fórmula de la distancia. (2011, February 20).https://es.khanacademy.org/math/geometry/hs-geo-analytic-geometry/hs-geo-distance-and-midpoints/v/distance-formula}
   {https://es.khanacademy.org/math/geometry/hs-geo-analytic-geometry/hs-geo-distance-and-midpoints/v/distance-formula}
   \\
   \sección{Referencias del problema 2}\\
   \bibitem{referencia citada}
   \\
   2-Repaso del discriminante (artículo). (s/f). Khan Academy. Recuperado el 22 de noviembre de 2023, de https://es.khanacademy.org/math/algebra/x2f8bb11595b61c86:quadratic-functions-equations/x2f8bb11595b61c86:quadratic-formula-a1/a/discriminant-review
   \bibitem{referencia citada}
  \\
   3-Introducción. (2016, abril 20). Portal Académico del CCH. https://e1.portalacademico.cch.unam.mx/alumno/matematicas2/unidad1/formulageneral/introduccion
   \bibitem{referencia citada}
   \\
   4-Comprender la fórmula de la cuadrática (artículo). (s/f). Khan Academy. Recuperado el 21 de noviembre de 2023, de https://es.khanacademy.org/math/algebra/x2f8bb11595b61c86:quadratic-functions-equations/x2f8bb11595b61c86:quadratic-formula-a1/a/quadratic-formula-explained-article
   \\
   \sección{Referencias del problema 3}\\
   \bibitem{referencia citada} 
   \\
   5- Fórmula de la distancia (artículo). -(s/f). Khan Academy. Recuperado el 21 de noviembre de 2023, de https://es.khanacademy.org/math/geometry/hs-geo-analytic-geometry/hs-geo-distance-and-midpoints/a/distance-formula.
 
   
   \bibitem[referencia citada]
   \\
   6-Fórmula de la distancia. (2011, February 20).https://es.khanacademy.org/math/geometry/hs-geo-analytic-geometry/hs-geo-distance-and-midpoints/v/distance-formula}
   \\
   7{https://es.khanacademy.org/math/geometry/hs-geo-analytic-geometry/hs-geo-distance-and-midpoints/v/distance-formula}
   \\
   \sección{Referencias del problema 4}\\
\bibitem[referencia citada]

  \\
  8-Menezes, P. M. S. (2018). Binary to Decimal Conversion Algorithm. International Journal of Computer Science and Information Security (IJCSIS), 16(3), 104-108. DOI: 10.5281/zenodo.1217082

 \bibitem[referencia citada]
 \\
  9-Li, K. H. (2012). Binary to Decimal Conversion Algorithm and Its Application. En Proceedings of the 2012 International Conference on Cyber-Enabled Distributed Computing and Knowledge Discovery (CyberC) (pp. 202-205). DOI: 10.1109/CyberC.2012.52
   \bibitem[referencia citada]
   \\
   \sección{Referencias del problema 5}\\
  \\
   10-El, P. (2016, diciembre 28). Sistemas de 
   \url{https://matemelga.wordpress.com/2016/12/28/sistemas-de-numeracion-posicionales/} \\
   
   \bibitem[referencia citada] 
   \\
   11-El, P. (2016, diciembre 28). Sistemas de 
   \url{https://matemelga.wordpress.com/2016/12/28/sistemas-de-numeracion-posicionales/} \\
   
   \bibitem[referencia citada]
   \\
   12-(S/f). Java.com. Recuperado el 22 de noviembre de 2023. \url{https://www.java.com/es/download/help/whatis_java.html}\\

\end{thebibliography}\vspace*{-8pt}


\begin{IEEEbiography}{Chávez Atancio Yael Antonio}{\,}Es un estudiante de la ingeniería en Tecnologías de la Información y Comunicaciones con una pasión por los E-Sports, escuchar y crear musica de todos los generos. Nacido y criado en Mixquiahuala de Juaréz Hidalgo, desde muy pequeño quiso tener una computadora para poder diseñar presentaciones y hacer trabajos simples en Word y Paint. Aunque parezca que no tiene intereses, tiene intereses distintos, Yael encuentra la paz y tranquilidad escuchando musica es Melomano. Esto le ha enseñado la importacia de aprender de sus errores y no carese. El objetivo de Yael es terminar la carrrera  en ITICs es para que le digan Inge aunque aun no sabe que rama estudiar, pronto la encontrara.
%\vadjust{\vfill\pagebreak}
\end{IEEEbiography}

\begin{IEEEbiography}{Hernandez Martinez Brayan}{\,} Es un estudiante del instituto ITSOEH de la carrea TICS, le gusta mucho la musica y tambien le gusta mucho hacer musica, siempre le gustaron las tecnologias desde pequeño se dedico a su investigacion. Nacido en Mixquiahuala de Juarez, criado en Tlaxcoapan Hidalgo, su objetivo es terminar sus estudios.
    %\vadjust{\vfill\pagebreak}
    \end{IEEEbiography}
\begin{IEEEbiography}{Cruz Martinez Alejandro}{\,}Es un estudiante del instituto ITSOEH de la carrea TICS, le gusta mucho salir de paseo y la musica de banda, siempre le gusto diseñar paginas web desde la secundaria. Nacio en Tlaxcoapan Hidalgo, fue criado en chilcuautla Hidalgo,su objetivo es tener una empresa.

    \begin{IEEEbiography}{Maldonado Olguin Irving}{\,} Estudiante del Instituto Tecnológico Superior del Estado de Hidalgo, en la carrera de Tics, le apasiona el ejercicio y la música, además de interesarle las tecnologías y desea aprender más. Nacido en el municipio de Mixquiahuala de Juárez, Hgo. Siendo una persona bastante tranquila, es bastante amable.
        %\vadjust{\vfill\pagebreak}
        \end{IEEEbiography}
    
\end{IEEEbiography}
\end{document}

